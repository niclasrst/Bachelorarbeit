%************************************************
\chapter{The main theorem}
%************************************************

To keep the main goal of this work in sight, we state the theorem we want to proof (or at least sketch some aspects) again here.

\begin{theorem*}
    A finitely generated right-angled Coxeter group \(W\) has a finite index subgroup \(W'\) such that \(W'\) is residually finite and rationally solvable.
\end{theorem*}

% Following the proof in Agol's work, we take a convex subset \(C \subset \R^n\), homeomorphic to the Tits cone \(WC\) of the Coxeter group \(W\).
% By passing to the quotient \(C/G\)...

For the following, we consider the Tits cone \(WC\) corresponding to \(W\), living in \(\R^n\) for \(n = \abs{I}\).
By the discusssion of last chapter, we know that by taking the quotient \(WC/W\), we are left with the fundamental chamber \(C\).
In the first section, we want to produce a manifold cover \(C' \to C\) of the fundamental chamber.
This cover will be induced by a finite index subgroup in the right-angled Coxeter group \(W\).

% The first step in the proof is to produce a manifold cover \(D' \to D\), which will be induced by a finite index subgroup \(W'\) in the right-angled Coxeter group \(W\).

\section{Constructing the manifold cover}

Consider the abelianization \(W_{\text{ab}}\) of \(W\), which is isomorphic to \((\Z/2\Z)^{n}\).
Now, the abelianization yields a homomorphism \(\alpha : W \to W_{\text{ab}}\), whose kernel denoted \(\ker\alpha\), we turn our attention to. % will be interested in.
First note that by the first homomorphism theorem, \(\ker\alpha\) has finite index in \(W\), since
\[\abs{\faktor{W}{\ker\alpha}} = \abs{\faktor{\Z}{2\Z}}^{n} = 2^{n} < \infty.\]
Here we use the fact that \(W\) is finitely generated, whence \(n = \abs{I} = \abs{S} < \infty\).

Furthermore, note that for each \(J \subset I\) with \(W_J\) finite, we have an isomorphism between \(W_J\) and \((\Z/2\Z)^{\abs{J}}\).
Thus, the restriction of \(\alpha\) to each such subgroup \(\alpha\vert_{W_J}\) is an injcetive homomorphism.
We now use that in the right-angled case, the isotropy subgroups of codimension-\(k\) faces are all of this form.
This can be seen, as the isotropy subgroup of a codimension-\(k\) face \(F\) is generated by the reflections in the \(k\) codim.-\(1\) faces, whose intersection forms \(F\).
As alle these codim.-\(1\) faces meet at angle \(\frac{\pi}{2}\), the generators commute paiwise.
Therefore, all isotropy subgroups inject into the abelianization of \(W\).
This implies that the intersection of an isotropy subgroup with the kernel of \(\alpha\) is trivial and consequently no isotropy group is contained in the kernel \(\ker\alpha\).
Since finite subgroups are contained in isotropy subgroups, the kernel \(\ker\alpha\) acts freely on the Tits cone \(WC\) of \(W\).

In particular, by \Cref{thm:stabilizer} the action of \(W\) on its Tits cone is also properly discontinuous and by \Cref{thm:convexity} the Tits cone is also a convex cone, implying that it has trivial fundamental group.
Having all this information, we are able to apply \Cref{lem:covering} and obtain the covering
\[WC \;\longrightarrow\; \faktor{WC}{\ker\alpha} \qquad x \mapsto \text{Orb}_{\ker\alpha}(x).\]
% As the Tits cone \(WC\) is a subspace of \(\R^{n}\), the definition of a covering map implies that the quotient \(WC/\ker\alpha\) is a manifold.
Using the local homeomorphism property of a covering map and the fact that the Tits cone \(WC\) is a subspace of \(\R^n\), we conclude that the quotient \(WC/\ker\alpha\) is indeed a manifold.

Thus, we set \(W' := \ker\alpha\) and now have to show that the map \(WC/W' \to WC/W\) is a covering map.
As the fundamental chamber \(C\) is homeomorphic to \(WC/W\), we have a cover as desired.


\section{Some orbifold theory - a sketch}