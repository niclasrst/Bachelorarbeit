%************************************************
\chapter{The main theorem}
%************************************************

To keep the main goal of this work in sight, we state the theorem we want to proof (or at least sketch some aspects) again here.

\begin{theorem*}
    A finitely generated right-angled Coxeter group \(W\) has a finite index subgroup \(W'\) such that \(W'\) is residually finite and rationally solvable.
\end{theorem*}

% Following the proof in Agol's work, we take a convex subset \(C \subset \R^n\), homeomorphic to the Tits cone \(WC\) of the Coxeter group \(W\).
% By passing to the quotient \(C/G\)...

For the following, we consider the Tits cone \(WC\) corresponding to \(W\), living in \(\R^n\) for \(n = \abs{I}\).
Before continuing, we want to shortly mention that the fundamental chamber \(C\) has a natural orbifold structure as the quotient \(WC/W\).
This will be covered later on in slightly greater detail.
% By the discusssion of last chapter, we know that by taking the quotient \(WC/W\), we are left with the fundamental chamber \(C\).
In the first section, we want to produce a manifold cover \(C' \to C\) of the fundamental chamber.
This cover will be induced by a finite index subgroup in the right-angled Coxeter group \(W\).


\section{Constructing the manifold cover}

Consider the abelianization \(W_{\text{ab}}\) of \(W\), which is isomorphic to \((\Z/2\Z)^{n}\).
Now, the abelianization yields a homomorphism \(\alpha : W \to W_{\text{ab}}\) to whose kernel, denoted \(\ker\alpha\), we turn our attention to. % will be interested in.
First note that by the first homomorphism theorem, \(\ker\alpha\) has finite index in \(W\), since
\[\abs{\faktor{W}{\ker\alpha}} = \abs{\faktor{\Z}{2\Z}}^{n} = 2^{n} < \infty.\]
Here we use the fact that \(W\) is finitely generated, whence \(n = \abs{I} = \abs{S} < \infty\).

Furthermore, note that for each \(J \subset I\) with \(W_J\) finite, we have an isomorphism between \(W_J\) and \((\Z/2\Z)^{\abs{J}}\).
Thus, the restriction of \(\alpha\) to each such subgroup \(\alpha\vert_{W_J}\) is an injective homomorphism.
We now use that in the right-angled case, the isotropy subgroups of codimension-\(k\) faces are all of this form.
This can be seen, as the isotropy subgroup of a codimension-\(k\) face \(F\) is generated by the reflections in the \(k\) codim.-\(1\) faces, whose intersection forms \(F\).
As all these codim.-\(1\) faces meet at angle \(\frac{\pi}{2}\), the generators commute paiwise.
Therefore, all isotropy subgroups inject into the abelianization of \(W\).
This implies that the intersection of an isotropy subgroup with the kernel of \(\alpha\) is trivial and consequently no isotropy group is contained in the kernel \(\ker\alpha\).
Since finite subgroups are contained in isotropy subgroups, the kernel \(\ker\alpha\) acts freely on the Tits cone \(WC\) corresponding to \(W\).

In particular, by \Cref{thm:stabilizer} the action of \(W\) on its Tits cone is properly discontinuous and by \Cref{thm:convexity} the Tits cone is also a convex cone, implying that it has trivial fundamental group.
Having all this information, we are able to apply \Cref{lem:covering} to obtain the covering
\[WC \;\longrightarrow\; \faktor{WC}{\ker\alpha} \qquad x \mapsto \text{Orb}_{\ker\alpha}(x).\]
% As the Tits cone \(WC\) is a subspace of \(\R^{n}\), the definition of a covering map implies that the quotient \(WC/\ker\alpha\) is a manifold.
Using the local homeomorphism property of a covering map and the fact that the Tits cone \(WC\) is a subspace of \(\R^n\), we conclude that the quotient \(WC/\ker\alpha\) is indeed a manifold.

% Thus, we set \(W' := \ker\alpha\) and now have to show that the map \(WC/W' \to WC/W\) is a covering map.
% As the fundamental chamber \(C\) is homeomorphic to \(WC/W\), we have a cover as desired.
% The insights presented above suggest that we set \(W' := \ker\alpha\).
% To justify this, we need to show that the map \(WC/W' \to WC/W\) is a covering map, since then we have a manifold cover of the fundamental chamber \(C\), as it is homeomorphic to \(WC/W\).
% But this is clearly satisfied, as \(W'\) is a subgroup of \(W\) and thus, \(WC \to WC/W'\) factors through \(WC/W' \to WC/W\)


\section{Some orbifold theory}

We want to elaborate more on the orbifold structure of the fundamental chamber \(C\).
Let us start by giving a formal definition of an orbifold.
We break this definition down into more pieces, starting with local models.

\begin{definition}
    A \emph{local model} is a pair \((\widetilde{U}, \Gamma)\), where \(\widetilde{U} \subset \R^n\) is open and \(\Gamma\) is a finite subgroup of the group of diffeomorphisms of \(\widetilde{U}\), denoted \emph{diffeo}\((\widetilde{U})\), acting on \(\widetilde{U}\).
    By abusing notation, we will sometimes say that the quotient \(U = \faktor{\widetilde{U}}{\Gamma}\;\) is a local model.
\end{definition}

Now that we defined a local structure of an orbifold, we want to translate between these local models.
This is being made precise by the next definition.

\begin{definition}
    An \emph{orbifold map} between local models \((\widetilde{U}_i, \Gamma_i), (\widetilde{U}_j, \Gamma_j)\) is a pair of maps \((\widetilde{\psi}, \varphi)\).
    The map \(\widetilde{\psi} : \widetilde{U}_i \to \widetilde{U}_j\) is smooth and \(\varphi : \Gamma_i \to \Gamma_j\) a homomorphism of groups, such that the map \(\widetilde{\psi}\) is \(\varphi\)-equivariant, meaning that for all \(g \in \Gamma_i\) and all \(\widetilde{x} \in \widetilde{U}_i\), \(\widetilde{\psi}(g\widetilde{x}) = \varphi(g)\widetilde{\psi}(\widetilde{x})\) holds.
    Then \(\widetilde{\psi}\) induces a map \(\psi : \faktor{\widetilde{U}_i}{\Gamma_i} \to \faktor{\widetilde{U}_j}{\Gamma_j}\), between the local models.
    When all three of these maps are injective, we call \(\psi\) a local isomorphism.
\end{definition}

Now that we have these local definitions, we `glue' them together, to obtain an orbifold.

\begin{definition}
    An \(n\)-dimensional (smooth) \emph{orbifold} \(Q\) is a pair \((X_Q, \mathcal{A})\).
    The space \(X_Q\) is a paracompact Hausdorff space, called the \emph{underlying space}.
    The set \(\mathcal{A}\) is called an \emph{orbifold atlas}, consisting of charts \((U_i, \phi_i)\), indexed by some set \(I\) and satisfying the following conditions:
    \begin{itemize}
        \item the \(U_i\) form an open cover of the underlying space \(X_Q\),
        \item for each \(U_i\) there exists a local model \(\faktor{\widetilde{U}_i}{\Gamma_i}\) with an homeomorphism \(\phi_i : U_i \to \faktor{\widetilde{U}_i}{\Gamma_i}\) and
        \item charts have to be compatible, meaning that for \(U_i \subset U_j\) the inclusion is a local isomorphism.
    \end{itemize}
\end{definition}

To sketch the connection between manifolds and orbifolds, let us mention one last thing.

\begin{definition}
    The \emph{local group} \(loc(x)\) of some \(x\) in a local model \(\faktor{\widetilde{U}}{\Gamma}\;\) is the isotropy group of any \(\widetilde{x}\) in \(\widetilde{U}\), getting projected onto \(x\).
    The \emph{singular locus} \(\Sigma (Q)\) of an orbifold \(Q\) consists of all points in the underlying space \(X_Q\) with non-trivial local group, i.e. \(\Sigma(Q) = \{x \in X_Q \;\vert\; loc(x) \neq \{1\}\}\).
\end{definition}

By this definition, we see that an orbifold with empty singular locus is just a manifold.
Furthermore, when thinking about an orbifold, we can just think about the underlying space and label each element in the singular locus by its local group.


\section{The cofinal cover}