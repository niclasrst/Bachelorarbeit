% ************************************************
\chapter{Introduction}\label{ch:introduction}
% ************************************************

The theory of \(3\)-dimensional manifolds is far richer and much more complex than that of the more familiar \(2\)-dimensional case of surfaces.
While a significant milestone was achieved in \(2002\) with the completion of the \(3\)-manifold classification by Grigori Perelman, many questions about their more detailed structure still remained open.
In his work on \(3\)-manifolds, kleinian groups and hyperbolic manifolds \cite{Thurston1982}, Thurston proposed \(24\) open questions that could guide future research in this area.
In his paper \cite{Agol2008}, Ian Agol tackled question \(18\) of Thurston, namely whether every hyperbolic \(3\)-manifold virtually fibers, i.e. has a finite-sheeted cover that fibers over \(S^1\).
Many authors gave reason to hope for a positive answer to this question, indeed Thurston himself wrote that `this dubious-sounding question seems to have a definite chance for a positive answer'.
Since the proposal of this question, many \(3\)-manifolds have been shown to virtually fiber but only few criteria for when a manifold virtually fibers were established.
In many cases, the cover that fibers over \(S^1\) was exhibited fairly explicit, for example in Walsh's work on two-bridge knots, Montesinos knots and link complements \cite{Walsh2004}.

Agol introduces a more abstract condition on the fundamental group of a \(3\)-manifold, called the RFRS property.
In the following of his work, he then proves that if an irreducible \(3\)-manifold has a fundamental group, satisfying the RFRS property, is virtually fibered.

Given this context, we allude to what this work is about and how it relates to the theory of \(3\)-manifolds.
We will be interested in one theorem in Agols' work \cite[Criteria for virtual fibering]{Agol2008}, stating that every right-angled Coxeter group virtually is RFRS.
This result is interesting, as the group of right-angled Coxeter groups is in some sense `nice enough' to satisfy the RFRS property, yet big enough to contain, or be related to, many other classes of groups.
One example are the surface groups, as they are subgroups of the reflection group in a right-angled pentagon and thus, subgroups of a right-angled Coxeter group.
Another example are the right-angled Artin groups, that are by definition closely related to right-angled Coxeter groups.
By a result of Michael Davis and Tadeusz Januszkiewicz \cite{Davis2000}, right-angled Artin groups are commensurable with right-angled Coxeter groups and thus, it can be shown that they are also RFRS.
This is stated as a corollary in Agols' work.
Thus, once it is clear that these classes of groups are RFRS, and we know of a manifold that its fundamental group is virtually in one of these classse, the manifold virtually fibers.

This work consists of two parts.
In the first half, i.e. in Chapter \(2\), we will start by introducing Coxeter groups, as well as right-angled Coxeter groups and will construct the representation, which consists basically of a vector space with a special bilinear form.
Then, we will see that the dual space is the natural space to look for a chamber in the representation of a Coxeter group and use it to define the fundamental chamber, as well as the Tits cone.
In Section \(2.3\) we will give an explicit example of the Tits cone, as it can be thought of quite geometrically.

\noindent
In the following sections, we will prove several important properties of Coxeter groups, starting with the proof that the representation of Coxeter groups is indeed a faithful representation.
We will further work out, that the so called fundamental chamber is a fundamental domain of the group action via the representation, as well as how stabilizers of points look like.
A key insight will be that a Coxeter group acts properly discontinuous on the interior of its corresponding Tits cone.
Finally, in the last section of Chapter \(2\), we will put together a toolbox, concerning Covering theory and connecting it to group actions, as this will be important in the proof of the main theorem of this work.

As a sort of guidance for the reader through the proof of the main theorem, covered in Chapter \(3\), we want to outline the ideas of the sections in it in the following as well.
First, we introduce the RFRS property on groups, which we already mentioned above.
In what follows, we will discuss what we have to show to verify this property for a group \(G\).

\noindent
Building onto the developed understanding of the RFRS property, clearly, we want to proceed by finding a candidate for the finite index subgroup \(W'\) of \(W\) that will satisfy the RFRS condition.
This candidate will induce a manifold cover of the fundamental chamber \(C\), corresponding to \(W\) and will be introduced in Section \(3.2\).

\noindent
Continuing from there on, the general goal is to translate from a cofinal sequence of manifold covers of the fundamental chamber \(C\), to a sequence of groups.
To achieve this, there is no way around some orbifold theory, as the fundamental chamber \(C\) carries a natural orbifold structure.
Thus, Section \(3.3\) will cover some aspects of the theory of orbifolds.

\noindent
In Section \(3.4\) we will use the developed tools of the theory of orbifolds, to construct a cofinal sequence of orbifold covers, beginning in the fundamental chamber \(C\).
This sequence will in particular induce a cofinal sequence of subgroups, which we want to be a sequence of subgroups of the finite index subgroup \(W'\).

\noindent
This will be dealt with in Section \(3.5\), where the aforementioned sequence will somewhat be forced to be a sequence of subgroups in \(W'\).
We will see that this new cofinal sequence of groups induces a sequence of manifold covers, starting in the cover established in Section \(3.2\).

\noindent
Up to then, we will have produced a cofinal sequence of subgroups living in \(W'\) and it remains to show that the rationally derived subgroup at step \(i\), is a subgroup of the following subgroup at step \(i+1\).
This will be the content of the last Section \(3.6\), where we will watch loops bouncing off the faces in an orbifold.
We will see `how' specifically loops behave in the orbifolds, given by polytopes \(P_i\), which will then allow us to deduce the main theorem. 