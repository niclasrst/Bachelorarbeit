% ***********************************************
\chapter{Preliminaries}\label{ch:preliminaries}
% ***********************************************

% ***********************************************
\section{Coxeter Groups}
% ***********************************************

First of all, we define the main object of interest we want to study.

\begin{definition}\label{def:CoxeterGroup}
    Let \(S\) be a set consisting of elements \(s_i\) indexed by an index set \(I\) with \(\abs{I} = n < \infty\).
    Let  \((m_{ij}) = M\) be a symmetrical matrix in \((\N\cup\{\infty\})^{n\times n}\), where \(m_{ii}=1\) for all \(i\) and \(m_{ij}\geq 2\) for \(i\neq j\).
    Define a group \(W\) via the following presentation:
    \begin{equation*}
        W := \groupp{S}{(s_is_j)^{m_{ij}}=1 \; \text{for all} \; i,j\in I}.
    \end{equation*}
    The pair \((W, S)\) is called a \emph{Coxeter System} and \(M\) is called the corresponding \emph{Coxeter Matrix}.
    A \emph{Coxeter group} is a group isomorphic to a group \(W\), corresponding to a Coxeter System \((W,S)\).
    It is generated by the set \(S\).
\end{definition}

In this work we will be particularly interested in a special class of Coxeter groups that we call right-angled.
They are defined as follows, by imposing significant constraints on the entries of the Coxeter matrix.

\begin{definition}
    A Coxeter System \((W, S)\) is \emph{right-angled} if, for all distinct \(i,j\in I\), the condition \(m_{ij}\in\{2,\infty\}\) is satisfied.
    In this context, the group \(W\) is then called a \emph{right-angled Coxeter group (RACG)}.
\end{definition}

We give some important examples of Coxeter groups as well as right-angled Coxeter groups.

\begin{example}
    \begin{enumerate}\label{ex:freeprod}
        \item Dihedral groups, \(D_{2m} \cong \groupp{s_1, s_2}{s_1^2 = s_2^2 = 1, (s_1s_2)^m = 1}\) are Coxeter groups for all \(m\in\N\).
        \item The triangle groups, \(\Delta(l,m,n) \cong \groupp{r,s,t}{r^2 = s^2 = t^2 = (rs)^l = (st)^m = (tr)^n = 1}\) \\
              with \(l, m\) and \(n\) integers greater or equal to \(2\) are Coxeter groups.
        \item The infinite Dihedral group, \(D_\infty \cong \groupp{s,t}{s^2=t^2=1}\) is a right-angled Coxeter group.
        \item The free product, \(\faktor{\Z}{2\Z}\ast\faktor{\Z}{2\Z}\ast\faktor{\Z}{2\Z} \cong \groupp{r,s,t}{r^2=s^2=t^2=1}\) is a right-angled Coxeter group.
              %\item Free products of a finite number of factors \(\faktor{\Z}{2\Z}\) are right-angled.
    \end{enumerate}
\end{example}

% % ***********************************************
% \subsection{Parabolic subgroups}
% % ***********************************************

% For a subset of the index set \(I\), we are able to construct subgroups of a given Coxeter group \(W\).
% They will be called parabolic subgroups and are defined as follows:
We can define a special type of subgroup, called a parabolic subgroup, within a Coxeter group \(W\).
These subgroups are constructed from a subset of the index set \(I\).
The definition is as follows:

\begin{definition}
    Let \((W,S)\) be a Coxeter System as above with finite index set \(I\), and \(J\) be a subset of the index set \(I\).
    The group \(W_J := \langle\{s_j \;\vert\; j\in J\}\rangle\), generated by the \(s_j\) in \(S\) is then called a \emph{parabolic subgroup} of \(W\).
    Moreover, we call any conjugate of \(W_J\) a parabolic subgroup as well.
\end{definition}

Once we have constructed the representation of Coxeter groups on a vector space as well as the Tits cone in the coming section, the parabolic subgroups will be a useful tool to form a deeper understanding of these objects.
We will extensively use them in sections \(2.4\) and \(2.5\).

% ***********************************************
\section{Representation of Coxeter groups}
% ***********************************************

Given a Coxeter System \((W,S)\), let \(V\) be a real vector space with basis \(\{e_1,\ldots,e_n\}\), where \(n=\abs{I}=\abs{S}\).
This provides a natural identification, \(GL_n(V)\cong GL_n(\R)\).
We define a bilinear form \(B_W\) on \(V\) as follows:
\begin{equation*}
    B_W := \begin{cases}
        -\cos \left(\frac{\pi}{m_{ij}}\right) & ,\; m_{ij}<\infty \\
        -1                                    & ,\; m_{ij}=\infty
    \end{cases}.
\end{equation*}
By \Cref{def:CoxeterGroup}, it is assured that \(m_{ij}\geq 2\) for distinct \(i,j\), ensuring the cosine term is non-positive.
Consequently, we have \(B_W(e_i,e_j)\leq 0\) for distinct \(i,j\).
Furthermore, from \(m_{ii}=1\), it follows that \(B_W(e_i,e_i)=1\).
Using this bilinear form, we define hyperplanes with corresponding reflections for each basis element \(e_i\) as follows:
\[H_i := \{v\in V\;\vert\; B_W(e_i,v) = 0\}\;, \qquad \sigma_i : V\to V,\quad v\mapsto v - 2B_W(e_i,v)e_i.\]

\begin{theorem}\label{thm:repr}
    The map given by:
    \[\rho : W \to GL_n(V) \cong GL_n(\R), \quad s_i \mapsto \sigma_i\]
    is an injective homomorphism and therefore a faithful representation of \(W\).
\end{theorem}
% TODO: evtl. ist der beweis zu nah am buch
Before we prove the homomorphism property, we want to recall:
A map \(\varphi: S\to G\) from a set \(S\) to a group \(G\) extends to a homomorphism \(\Hat{\varphi}:\groupp{S}{R}\to G\), if and only if the induced homomorphism \(\overline{\varphi}:F_S \to G\) from the free group over \(S\) satisfies \(\overline{\varphi}(r) = e_G\) for every \(r\in R\).
\begin{proof}
    Observe that \(\sigma_i^2 = id\) in \(GL_n(V)\) and thus, by applying the above to our situation we see that it suffices to show that the product \(\sigma_i\sigma_j\) has order \(m_{ij}\) in \(GL_n(V)\) for distinct \(i,j\in I\).
    To do so, consider the two-dimensional subspace \(V_{ij}\) spanned by \(e_i\) and \(e_j\) in \(V\).
    We take a general element \(v = \lambda e_i + \mu e_j\), \(\lambda,\mu\in\R\) in \(V_{ij}\) and distinguish the two cases in the definition of \(B_W\):
    \begin{itemize}
        \item[1)] \(m_{ij}<\infty\): In this case \(B_W\) is positive definite, since for \(v\neq 0\)
              \begin{align*}
                  B_W(v,v) & = \lambda^2 - 2\lambda\mu\cos\left(\frac{\pi}{m_{ij}}\right) + \mu^2
                  = \left( \lambda - \mu\cos\left( \frac{\pi}{m_{ij}} \right) \right)^2 + \mu^2\sin^2\left( \frac{\pi}{m_{ij}} \right) > 0.
              \end{align*}
              Up to a change of basis, we identify \((V_{ij}, B_W\vert_{V_{ij}})\) with the euclidean plane \((\R^2, \langle\cdot, \cdot\rangle)\).
              Now \(\sigma_i\) and \(\sigma_j\) act on \(V_{ij}\) by orthogonal reflections in the hyperplanes \(H_i, H_j\) intersected with \(V_{ij}\).
              We take a look at the inner product of \(e_i, e_j\)
              \[B_W(e_i, e_j) = -\cos\left(\frac{\pi}{m_{ij}}\right) = \cos\left(\pi - \frac{\pi}{m_{ij}}\right)\]
              and obtain that the angle between them in \(V_{ij}\) is given by \(\pi - \frac{\pi}{m_{ij}}\).
              Thus, the angle between the reflecting lines is \(\frac{\pi}{m_{ij}}\) and the product \(\sigma_i\sigma_j\) turns out to be a rotation by \(\frac{2\pi}{m_{ij}}\), showing that \(\sigma_i\sigma_j\) has order \(m_{ij}\) in the subspace \(V_{ij}\).
              And since \(\sigma_i\sigma_j\) fixes the orthogonal complement of \(V_{ij}\) by definition of the \(\sigma_i\), it has order \(m_{ij}\) on the whole vector space \(V\).

        \item[2)] \(m_{ij} = \infty:\;\) By the following, we now have to deal with a non-positive definite form:
              \begin{align*}
                  B_W(v,v) = \lambda^2 - 2\lambda\mu + \mu^2 = (\lambda - \mu)^2 \geq 0.
              \end{align*}
              Indeed, we can only expect it to be positive semidefinite on \(V_{ij}\).
              Using the calculation
              \begin{align*}
                  (\sigma_i\sigma_j)(e_i) = \sigma_i (e_i + 2e_j) = e_i + 2(e_i + e_j),
              \end{align*}
              together with an induction argument, we get that \((\sigma_i\sigma_j)^n(e_i) = e_i + 2n(e_i + e_j)\).
              Therefore, the concatenation has infinite order on \(V_{ij}\) and in particular on the whole of \(V\).
    \end{itemize}
    % Moreover, we see that two-generator subgroups of \(W\) are dihedral, either of order \(2m_{ij}\) or infinite order.
    This proves that \(\rho\) extends to a homomorphism.
    It remains to show the injectivity of \(\rho\).
    This will be a consequence of a bigger result in section \(2.4\), see \Cref{cor:faithful}.
\end{proof}

\begin{remark}
    It is worth to note that in the above proof, we have seen that two-generator subgroups of Coxeter groups are dihedral.
    Either of order \(2m_{ij}\) or infinite order.
\end{remark}

% We will prove the injectivity of \(\rho\) later. % [x]: Make this part of the above proof and say where precisely it will be proved.
We want to extend the action of \(W\) to the dual of the vector space \(V\).
This is achieved by acting on \(V^*\) via the dual representation of \(\rho\), defined by
% The group \(W\) will act on the dual space \(V^*\) via the dual of this representation, and we define the dual representation \(\rho^*\) as follows
\[\rho^* : W \to GL_n(V^*),\quad w \mapsto (\rho^*(w)(\varphi))(v) = \varphi(\rho(w^{-1})(v)), \quad w\in W, \varphi\in V^*, v\in V.\]
Notation wise, we will simply write \(w(v)\), when \(w \in W\) acts on \(v \in V\) via \(\rho(w)(v)\).
Similarly, we write \(w(\varphi)\) when we mean that \(w \in W\) acts on some element \(\varphi \in V^*\) of the dual space, via the dual representation \(\rho^*(w)(\varphi)\).
% Similar to the definitions of hyperplanes and reflections in \(V\), we want to give a definition for these notions in the dual space \(V^*\).
As in the case of the vector space \(V\), we want to give a definition for the notion of a hyperplane with corresponding reflection in the dual space \(V^*\) as well.
By dual hyperplane, we mean a subspace \(H_i^* := \{\varphi \in V^* \;\vert\; \varphi (e_i) = 0\}\), and the corresponding dual reflections will be a map from \(V^*\) to \(V^*\), given by:
\[\sigma_i^* : V^* \to V^*,\quad \varphi \mapsto \varphi \circ \sigma_i = \varphi - 2B_W(e_i,\;\cdot\;)\varphi(e_i).\]

% \begin{remark}
%     Notation wise, we will simply write \(w(v)\), when \(w\in W\) acts on \(v\in V\) via \(\rho(w)(v)\).
%     Similarly, \(w(\varphi)\) when \(w\in W\) acts on some \(\varphi\in V^*\) via the dual representation \(\rho^*(w)(\varphi)\).
% \end{remark}

To further explore Coxeter groups and their action via this representation, we need some more notation.
In particular, we want a so-called \emph{chamber}.
This should be thought of as a cone over a polytope with finitely many faces such that the reflections in its codimension one faces correspond to the generators of \(W\) under the representation.

\begin{definition}\label{def:chamber}
    The fundamental chamber \(C\) of the dual representation is the set, given by
    \[C := \{\varphi \in V^* \;\vert\; \varphi (e_i) \geq 0 \;\forall i\in I\} \subset V^*.\]
\end{definition}

Denote by \(\{e_1^*,\ldots, e_n^*\}\) the dual basis of \(V^*\) corresponding to the standard basis \(\{e_1,\ldots, e_n\}\) of \(V\).
Then we calculate, using the \(\sigma_i^*\) from above:
\begin{equation*}
    \sigma_i^*(e_j^*) = e_j^* - 2B_W(e_i,\cdot)e_j^*(e_i) =
    \begin{cases}
        e_j^*                   & \text{for } i\neq j \\
        e_j^* - 2B_W(e_j,\cdot) & \text{for } i=j
    \end{cases},
\end{equation*}
which implies that each reflection \(\sigma_i^*\) fixes all the hyperplanes \(H_j^*\), for distinct indices \(i\) and \(j\).
Moreover, note that the fundamental chamber can be written of the form % TODO: rework the following paragraph --
\[C = \bigcap_{i\in I}\{\varphi\in V^*\;\vert\; \varphi(e_i)\geq 0\} = \bigcap_{i\in I} (H_i^* \cup \{\varphi\in V^*\;\vert\; \varphi(e_i)> 0\}),\] % [x]: this is now right
where we observe that the \(H_i^*\) form the pairwise distinct codimension one faces of the chamber \(C\).
The open halfspaces \(\{\varphi \in V^* \;\vert\; \varphi(e_i) > 0\}\) in the latter term will be called \(A_i^*\) and using these, we define the open fundamental chamber to be the intersection of the open halfspaces:
\[\text{int}(C) = \mathring{C} = \bigcap_{i \in I} A_i^*.\] % --
As mentioned above, we want to study the action of our Coxeter group via the dual representation, acting by reflection in the faces \(H_i^*\).
However, in general the translates of the chamber under the group action won't cover the whole of \(V^*\), which motivates the following definition:

\begin{definition}
    The Tits cone is the union of all \(W\)-translates of the chamber, \(WC := \underset{w \in W}{\bigcup} wC \subset V^*\).
\end{definition}

As the name suggests, the fundamental chamber is a fundamental domain for the action of \(W\) on the Tits cone \(WC\) under the dual representation \(\rho^*\).
This will be proved in \Cref{thm:funddomain}.
While the formal defintion of the Tits cone provides a rigorous foundation, it is not very insightful from a geometric perspective.
% The formal definition of the Tits cone is not very insightful.
As one can think about the Tits cone quite geometrically, especially in low dimensions, we will take a closer look at an explicit example in the following section.
Before doing so, we end this section with the following remark.

\begin{remark}
    One may ask why we transport everything to the dual space, instead of working in the standard representation \(\rho\).
    For this, consider the infinite dihedral group \(D_\infty \cong \groupp{s, t}{s^2 = t^2 = 1}\).
    We fix a basis \(\{e_1, e_2\}\) of \(V\) and obtain that the bilinear form in this basis is given by the matrix
    \begin{equation*}
        B_W = \begin{pmatrix}
            1  & -1 \\
            -1 & 1
        \end{pmatrix}.
    \end{equation*}
    We observe that \(H_1 = \text{span}\{e_1 + e_2\} = H_2\), which implies that \(\sigma_1\) and \(\sigma_2\) fix the same hyperplane, despite having different \((-1)\)-Eigenspaces (namely the span of \(e_1\), resp. \(e_2\)).
    Therefore, in general, working in the standard representation won't result in a chamber, giving rise to the existence of the Tits cone.
    Now, passing to the dual space \(V^*\) by fixing the dual basis \(\{e_1^*, e_2^*\}\), consider the dual reflections \(\sigma_i^*\) as discussed before.
    By the more general calculation earlier, we obtain
    \[\sigma_1^*(H_2^*) = H_2^* \quad\text{and}\quad \sigma_2^*(H_1^*) = H_1^*.\]
    And furthermore, we note that for all \(i,j\in\{1,2\}\)
    \[\sigma_i^*(B_W(e_j,\cdot)) = B_W(e_j,\cdot) - 2B_W(e_i,\cdot)B_W(e_j,e_i) = - B_W(e_j,\cdot),\]
    using that \(B_W(e_j,\cdot) = -B_W(e_i,\cdot)\).
    This shows that both dual reflections have the same \((-1)\)-Eigenspace, but fix different hyperplanes (\ie, have different \((+1)\)-Eigenspaces), resulting in a chamber as wished.
\end{remark}


% ***********************************************
\section{The Tits cone - An example} % TODO: add picture
% ***********************************************

As an example we take a closer look at the free product \(W \cong\groupp{r,s,t}{r^2=s^2=t^2=1}\) from \Cref{ex:freeprod}.
We fix the basis \(\{e_1,e_2,e_3\}\) and identify \(V\) with \(\R^3\).
In this basis, the bilinear form \(B_W\) is given by the matrix
\begin{equation*}
    B_W =
    \begin{pmatrix}
        1  & -1 & -1 \\
        -1 & 1  & -1 \\
        -1 & -1 & 1
    \end{pmatrix}.
\end{equation*}
By the spectral theorem we find a basis of orthonormal Eigenvectors, in which \(B_W\) is a diagonal matrix with its eigenvalues as entries.
Using the Gram-Schmidt procedure, we get
\begin{equation*}
    \begin{pmatrix}
        -\frac{1}{\sqrt{2}} & 0                  & \frac{1}{\sqrt{2}}  \\
        -\frac{1}{\sqrt{6}} & \frac{2}{\sqrt{6}} & -\frac{1}{\sqrt{6}} \\
        \frac{1}{\sqrt{3}}  & \frac{1}{\sqrt{3}} & \frac{1}{\sqrt{3}}
    \end{pmatrix} \cdot
    \begin{pmatrix}
        1  & -1 & -1 \\
        -1 & 1  & -1 \\
        -1 & -1 & 1
    \end{pmatrix} \cdot
    \begin{pmatrix}
        -\frac{1}{\sqrt{2}} & -\frac{1}{\sqrt{6}} & \frac{1}{\sqrt{3}} \\
        0                   & \frac{2}{\sqrt{6}}  & \frac{1}{\sqrt{3}} \\
        \frac{1}{\sqrt{2}}  & -\frac{1}{\sqrt{6}} & \frac{1}{\sqrt{3}}
    \end{pmatrix} =
    \begin{pmatrix}
        2 & 0 & 0  \\
        0 & 2 & 0  \\
        0 & 0 & -1
    \end{pmatrix}.
\end{equation*}
In other words, we have \(V^T B_W V = D\) with \(V\in O(n)\) and \(D = \text{diag}(2,2,-1)\).
Now, since we have a diagonal matrix, we can multiply the entries by squares, since the resulting matrix will be congruent to the given one:

Let \(A = \text{diag}(\mu_1,\ldots,\mu_n)\in\R^{n\times n}\),
\begin{equation*}
    S=\text{diag}(\lambda_1,\ldots, \lambda_n)\in(\R\setminus\{0\})^{n\times n} \implies S^TAS = \text{diag}(\lambda_1^2\mu_1,\ldots,\lambda_n^2\mu_n).
\end{equation*}
To apply this and further transform our matrix \(D\), define the invertible matrix \(T\) as follows
\[T = \begin{pmatrix} \frac{1}{\sqrt{2}} & 0 & 0 \\ 0 & \frac{1}{\sqrt{2}} & 0 \\ 0 & 0 & 1 \end{pmatrix},\]
which then implies \(\Tilde{D} := T(V^T B_W V)T = \text{diag}(1,1,-1)\).
The images of the basis vectors \(\{e_1,e_2,e_3\}\) are given by the three columns of the matrix \(TV^T\), namely:
\begin{equation*}
    \Tilde{e}_1 = \begin{pmatrix} -\frac{1}{2} \\ - \frac{\sqrt{3}}{6} \\ \frac{1}{\sqrt{3}} \end{pmatrix},\;
    \Tilde{e}_2 = \begin{pmatrix} 0 \\ \frac{1}{\sqrt{3}} \\ \frac{1}{\sqrt{3}} \end{pmatrix} \text{ and }
    \Tilde{e}_3 = \begin{pmatrix} \frac{1}{2} \\ -\frac{\sqrt{3}}{6} \\ \frac{1}{\sqrt{3}} \end{pmatrix}.
\end{equation*}
Note that we have \(V\cong \R^3\), equipped with the inner product
\[\langle x, y \rangle_{2,1} := x^T\Tilde{D}y = x_1y_1 + x_2y_2 - x_3y_3.\]
Since \(\langle \Tilde{e}_i, \Tilde{e}_i \rangle_{2,1} = 0\) for all \(i\in \{1,2,3\}\), we have that these three vectors span an ideal triangle in the hyperboloid model of \(\HH^2\), given by \(\langle x, y \rangle_{2,1} = -1\).
One gets the ideal triangle by intersecting the hyperboloid with the hyperplanes spanned by each two of the \(\Tilde{e}_i\) (they will only intersect in the surrounding cone of the hyperboloid).

Given these new coordinates under the transformation \(TV^T\), the Tits cone will be given by \(x_1^2 + x_2^2 - x_3^2 < 0\) union the images of \(\Tilde{e}_1, \Tilde{e}_2\) and \(\Tilde{e}_3\) under the reflection in the sides of the chamber \ie, in the sides of the ideal triangle. % TODO: under the transformation TV^t
Moreover, we get a subgroup of \(O(2,1)_+\) generated by the three matrices
\begin{equation*}
    \begin{pmatrix} 1 & 0 & 0 \\ 0 & -\frac{5}{3} & -\frac{4}{3} \\ 0 & \frac{4}{3} & \frac{5}{3} \end{pmatrix},\;
    \begin{pmatrix} -1 & \frac{2}{\sqrt{3}} & -\frac{2}{\sqrt{3}} \\ \frac{2}{\sqrt{3}} & \frac{1}{3} & \frac{2}{3} \\ \frac{2}{\sqrt{3}} & -\frac{2}{3} & \frac{5}{3} \end{pmatrix}
    \text{ and }
    \begin{pmatrix} -1 & -\frac{2}{\sqrt{3}} & \frac{2}{\sqrt{3}} \\ -\frac{2}{\sqrt{3}} & \frac{1}{3} & \frac{2}{3} \\ -\frac{2}{\sqrt{3}} & -\frac{2}{3} & \frac{5}{3} \end{pmatrix}.
\end{equation*}
Each of the above matrices corresponds to one of the generators \(r, s\) and \(t\) of the right angled Coxeter group \(W\) under the transformation \(TV^T\), constructed above.


% ***********************************************
\section{The word metric and the faithful representation}
% ***********************************************

Recall that for any finitely generated group \(W = \langle S \rangle\), its Cayley graph induces a metric on \(W\), relative to the generating set \(S\).
We call it the \emph{word metric} of \(W\) relative to \(S\) and denote it by \(d_S\).
Also note that the word metric is left-invariant, meaning that for group elements \(u,v,w\in W\), we have the equality \(d_S(uv,uw) = d_S(v,w)\).
Now, define the \emph{word length} of an element \(w\in W\) to be \(\ell(w) := d_S(w, 1_W)\), the distance of an element to the neutral element of the group.
Note that \(\ell(w) = 0\) if and only if \(w = 1_W\).

\begin{lemma}
    We collect some properties of the length function we will use later on.
    \begin{enumerate}
        \item \(\forall w\in W:\; \ell(w) = \ell(w^{-1})\)
        \item \(\forall s\in S:\; \ell(s) = 1\) and \(\ell(w) = 1 \iff w\in S^{\pm 1}\)
        \item \(\forall v,w\in W:\; \ell(vw)\leq\ell(v)+\ell(w)\)
        \item \(\forall v,w\in W:\; \ell(v)-\ell(w)\leq\ell(vw)\)
        \item \(\forall w\in W, s\in S^{\pm 1}:\; \ell(w)-1\leq\ell(ws)\leq\ell(w)+1\)
    \end{enumerate}
\end{lemma}
\begin{proof}
    All of the above statements follow from the fact that \(d_S\) is a left-invariant metric.
    % \begin{enumerate}
    %     \item For \(w\in W\), write the length of w as \(\ell(w) = d_S(w, e_W)\).
    %           By the left-invariance of \(d_S\), we have \(d_S(w, e_W) = d_S(e_W, w^{-1})\) and by symmetry of \(d_S\), \(\ell(w) = \ell(w^{-1})\) follows.
    %     \item This is true by definition.
    %     \item For \(u,v\in W\), using the triangle inequality of \(d_S\), we get: \[\ell(uv) = d_S(uv, e_W) \leq d_S(u, e_W) + d_S(v, e_W) = \ell(u) + \ell(w).\]
    %     \item For \(v,w\in W\) we have \(\ell(v)=\ell(vww^{-1})\leq\ell(vw) + \ell(w^{-1})\), by point 3 and thus, using point 1: \(\ell(v) - \ell(w) \leq \ell(vw)\).
    %     \item For \(w\in W,\; s\in S\), apply point 3 and point 4 to \(\ell(ws)\).
    % \end{enumerate}
\end{proof}

Coming back to Coxeter groups, by definition each generator \(s\in S\) has order 2 in \(W\).
Therefore, we can write every non-trivial element in \(W\) as a sequence of generators in \(S\).
Note that in this sequence there might be redundencies, so that the following definition makes sense. % [x]: Not the only thing that can go wrong. -> replaced repetitions with redundencies
We call an expression \(w = s_{i_1} \cdots s_{i_r}\) for \(i_1,\ldots, i_r\in I\) and \(r\in\N\) \emph{reduced}, if \(\ell(w) = r\), \ie \(\;w\) cannot be represented by a shorter word.
These reduced expressions have the caveat of not being unique by any means.

Given a parabolic subgroup \(W_J\) of a Coxeter group \(W\), it admits its own word metric with respect to \(J\subset I\).
Therefore each parabolic subgroup admits its own length function, which we denote by \(\ell_J(w)\) for words \(w\) in \(W_J\).
In the following we will make use of the general fact that we have \(\ell(w) \leq \ell_J(w)\) for all \(w\in W_J\).

The length function turns out to be an important and powerful tool in studying Coxeter groups.
Indeed, its role will be demonstrated in several of the forthcoming proofs, beginning with the following theorem which is a key step in proving faithfulness of our previously defined representation.
\begin{theorem}\label{thm:action}
    Let \(w\in W\) and \(s_i\in S\) for \(i\in I\).
    Then \(\ell(ws_i) > \ell(w)\) implies that \(w(e_i) > 0\).
\end{theorem}

We prove the theorem by induction on \(\ell(w)\) and by decomposing \(w\), we will see how \(w\) acts on \(V\).
% For this, we have to check some conditions and look at the action on basis elements.
Note that by writing \(w(e_i)\) for some \(w\in W\) and \(i\in I\), we mean that \(e_j^*(w(e_i)) \geq 0\) for all \(j\in I\).
Put in other words, this means that \(w(e_i)\) is contained in every \(\overline{A_j^*} = A_j^* \cup H_j^*\) for \(j\in I\).
\begin{proof}
    The base case is trivial, as \(\ell(w) = 0\) implies \(w = 1_W\) and thus \(w\) fixes every basis element.
    Therefore, assume \(w\) is non-trivial and in reduced form.
    We state the induction hypothesis.
    \IH{\(\ell(vs_i) > \ell(v)\) for some \(v \in W\) and \(s_i \in S\) implies that \(v(e_i) > 0\).}
    \Claim{1}{There is a \(j\in I\) such that \(s_i\neq s_j\) and \(\ell(ws_j) = \ell(w) - 1\).}
    \Claimproof{1}{
        Since \(w\) is in reduced form, write \(w = s_{i_1}\cdots s_{i_k}\) and set \(s_j = s_{i_k}\).
        This implies \[\ell(ws_i) > \ell(w) > \ell(w) - 1 = \ell(ws_j),\]
        and in particular \(s_j\neq s_i\).
        This proves the first Claim.
    }
    Consider the parabolic subgroup \(\langle s_i, s_j\rangle\leq W\) generated by these two distinct elements in \(S\).
    Denote this subgroup by \(W_J\), for \(J=\{i,j\}\subset I\).
    Our goal now is to decompose \(w\) into two parts, one living in the subgroup \(W_J\) and the other in its complement.
    For this, consider a specific subset of the coset \(wW_J\), given by
    \[A:=\{v\in W \;\vert\; vW_J = wW_J \;\text{ and }\; \ell(v) + \ell_J(v^{-1}w) = \ell(w)\}.\]
    By definition, \(w\in A\).
    With regard to the following, we choose \(v\in A\) such that its length \(\ell(v)\) is minimal.
    % Note that since \(vW_J = wW_J\), we have that \(W_J = v^{-1}wW_J\), implying that \(v^{-1}w\in W_J\).
    % In view of the comment above, set \(v_J = v^{-1}w \iff w = vv_J\), decomposing \(w\) as desired.
    Now set \(v_J = v^{-1}w\), which is equivalent to writing \(w = vv_J\), giving us a decomposition of \(w\) as desired.
    Due to this decomposition, our analysis of \(w\) now boils down to studying the separate actions of \(v\) and \(v_J\) on \(V\).
    First note that \(ws_j\) is contained in \(A\) as well, since \(s_j^{-1} = s_j\) and thus
    \[s_jw^{-1}w = s_j\in W_J \text{ and } \ell(ws_j) + \ell_J(s_j) = \ell(w) - 1 + 1 = \ell(w).\]
    By our choice of \(\ell(v)\) to be minimal, we see \(\ell(v)\leq \ell(ws_j) = \ell(w)-1\).
    Thus, we are almost set up to apply the induction hypothesis to \(v\) and \(s_i\).
    The last ingredient for this is the following
    \Claim{2}{For the lengths of \(vs_i\) and \(v\), we have the relation: \(\ell(vs_i) \geq \ell(v)\).}
    \Claimproof{2}{
        Assume towards contradiction: \(\ell(vs_i) < \ell(v)\), equivalently \(\ell(vs_i) = \ell(v) - 1\).
        Then:
        \begin{align*}
            \ell(w) & = \ell(vv_J) = \ell(vs_is_iv^{-1}w) \leq \ell(vs_i) + \ell(s_iv^{-1}w)        \\
                    & \leq \ell(v) - 1 + \ell_J(v^{-1}w) + 1 = \ell(v) + \ell_J(v^{-1}w) = \ell(w).
        \end{align*}
        This means equality holds throughout and in particular \(\ell(w) = \ell(vs_i) + \ell_J(s_iv^{-1}w)\).
        But this implies \(vs_i\) belongs to \(A\), contradicting the minimality of \(\ell(v)\).
    }
    Applying the induction hypothesis \textbf{(IH)} to \(v\) and \(s_i\) leaves us with \(v(e_i) > 0\).
    The exact same argument applied to \(v\) and \(s_j\) shows \(v(e_j) > 0\), so that we omit this here.
    Next, our goal is to study how \(v_J\) acts on \(V\).
    More precisely, how it acts on the basis element \(e_i\).
    \Claim{3}{For \(v_Js_i\) and \(v_J\) we have the relation: \(\ell_J(v_Js_i)\geq \ell(v)\).}
    \Claimproof{3}{
        Assume towards contradiction, that \(\ell_J(v_Js_i) < \ell(v_J)\) holds. Then:
        \begin{align*}
            \ell(ws_i) = \ell(vv_Js_i) \leq \ell(v) + \ell(v_Js_i) \leq \ell(v) + \ell_J(v_Js_i) < \ell(v) + \ell_J(v_J) = \ell(w),
        \end{align*}
        which contradicts the assumption of the theorem, that \(\ell(ws_i) > \ell(w)\).
    }
    Moreover, this shows that all reduced expressions of \(v_J\) in the parabolic subgroup \(W_J\) have to end in \(s_j\).
    Else we would have \(\ell(v_J) > \ell(v_J s_i)\), contradicting \emph{Claim 3}.
    It remains to show
    \Claim{4}{\(s_is_j\) maps \(e_i\) to a non-negative linear combination of \(e_i\) and \(e_j\).}
    \Claimproof{4}{
        Note that \(W_J\) is dihedral, either of order \(2m_{ij}\) or infinite order.
        Then, as \(v_J\) lies in \(W_J\), any reduced expression of \(v_J\) in \(W_J\) is an alternating product of \(s_i\) and \(s_j\) ending in \(s_j\).
        Now, distinguish by the order of the group \(W_J\):
        \begin{itemize}
            \item[1)] \(m_{ij} < \infty\): \(W_J\) is dihedral of order \(2m_{ij}\). %and its Cayley graph is a cycle of length \(2m_{ij}\).
                  Thus, any element is of length \(< m_{ij}\) and has a unique word corresponding to the edge labels in the Cayley graph.
                  Note that the maximum length of \(w\in W_J\) is precisely \(m_{ij}\) (as the Cayley graph is a cycle of length \(2m_{ij}\)).
                  The element of length \(m_{ij}\) is represented by the reduced expressions \(s_is_j\cdots s_j\) and \(s_js_i\cdots s_i\).
                  This implies that \(v_J\) has to have length smaller \(m_{ij}\), else it would have a reduced expression ending in \(s_i\), contradicting the above.
                  Then \(v_J\) is a product of less than \(\frac{m_{ij}}{2}\) terms \(s_is_j\), and each of the products \(s_is_j\) is a counterclockwise rotation about \(\frac{2\pi}{m_{ij}}\).
                  So \(v_J\) rotates \(e_i\) at maximum \(\pi - \frac{2\pi}{m_{ij}}\), which still lies inside the cone spanned by \(e_i\) and \(e_j\).
                  In particular the resulting vector is a non-negative linear combination. % TODO: eventuell s_j am anfang noch!

            \item[2)] \(m_{ij} = \infty\): In the case of the infinite dihedral group, we already have seen in the proof of \Cref{thm:repr} that \((s_is_j)^n(e_i) = e_i + 2n(e_i + e_j)\).
        \end{itemize}
        This proves the last Claim.
    }
    Since \(v_J\) and \(v\) both map \(e_i\) to a non-negative linear combination of \(e_i\) and \(e_j\), so does \(w\) due to its decomposition.
    Therefore, the proof is now complete.
\end{proof}

The result of the previous theorem readily extends to its converse, offering valuable insights as well.
We shall formally record this implication as a corollary.

\begin{corollary}\label{cor:lengthsmall}
    In the previous theorem, set \(w = \Tilde{w}s_i\) for \(i\in I\).
    Then \(\ell(\Tilde{w}s_i) < \ell(w)\) implies \(\ell(\Tilde{w}s_is_i) > \ell(\Tilde{w}s_i)\).
    Thus we have \((ws_i)(e_i) = -w(e_i) > 0\), or equivalently \(w(e_i) < 0\).
\end{corollary}

With \Cref{cor:lengthsmall} and \Cref{thm:action} in hand, we will finally be able to deduce the injectivity and thus faithfulness of our representation \(\rho\).

\begin{corollary}\label{cor:faithful}
    The homomorphism of \Cref{thm:repr} is injective and thus a faithful representation.
\end{corollary}
\begin{proof}
    Assume \(w\in\ker\{\rho\}\) non-trivial. % and \(w\neq e_W\) non-trivial.
    Then there is an \(i\in I\) with corresponding \(s_i\in S\), such that \(\ell(ws_i) < \ell(w)\).
    Now \Cref{cor:lengthsmall} implies \(w(e_i) < 0\), but \(w(e_i) = e_i > 0\) as \(\rho(w) = id_V\), which is a contradiction and thus, the statement holds.
\end{proof}


% ***********************************************
\section{The fundamental domain and Stabilizers}
% ***********************************************

In this section, we will prove that the fundamental chamber \(C\subset V^*\) is indeed a fundamental domain for the action of \(W\) on its Tits cone \(WC\).
Moreover, we will work out how stabilizers of points look like and then show that the Tits cone is really a convex cone in the dual space \(V^*\).

% \begin{remark}\label{rmk:chamber}
%     In previous definitions, we defined the fundamental chamber, hence the Tits cone to be closed.
%     We now want to distinguish between closed and open chamber and Tits cone.
%     As mentioned before, the fundamental chamber \(C\) can be written in the following way
%     \[C = \bigcap_{i\in I} H_i^* \cup \bigcap_{i\in I} \{\varphi\in V^*\;\vert\; \varphi(e_i) > 0\}.\]
%     Thus, we will call the latter sets \(A_i^* := \{\varphi\in V^*\;\vert\; \varphi(e_i) > 0\}\) and then define the interior of the chamber as the intersection of these open halfspaces in the dual space:
%     \[\text{int}(C) = \mathring{C} := \bigcap_{i\in I} A_i^*.\]
%     The interior of the Tits cone is then given by the union of translates of the open chamber \(\mathring{C}\):
%     \[\text{int}(WC) = W\mathring{C} := \bigcup_{i\in I} w(\mathring{C}).\]
%     We also want to mention, that each of the open halfspaces \(A_i^*\) gets permuted with its complementary open halfspace by the action of its corresponding generator \(s_i\) in \(S\):
%     \[s_i(A_i^*) = \{\varphi\in V^* \;\vert\; s_i(\varphi)(e_i) = \varphi(s_i(e_i)) = -\varphi(e_i)>0\} = \{\varphi\in V^* \;\vert\; \varphi(e_i)<0\}.\]
% \end{remark}

The following Lemma provides a criterion for when an element \(w\in W\) leaves the open fundamental chamber \(\mathring{C}\) in one of the open halfspaces \(A_i^*\).

\begin{lemma}\label{lem:translation}
    Let \(w\in W\) and \(i\in I\).
    Then the relation \(\ell(s_iw) > \ell(w)\) of \(s_i\) and \(w\) is equivalent to \(w(\mathring{C})\subset A_i^*\), \ie \(\;w\) leaving the open chamber inside the open halfspace, corresponding to \(s_i\).
\end{lemma}
\begin{proof}
    Note that \(\ell(s_iw) > \ell(w)\) is equivalent to \(\ell(w^{-1}s_i) > \ell(w^{-1})\), meaning that \(w^{-1}(e_i) > 0\) by \Cref{thm:action}.
    Now for \(\varphi\in \mathring{C}\), we have that \(w(\varphi)(e_i) = \varphi(w^{-1}(e_i))\) and therefore \(w(\varphi)\in A_i^*\).
    % \[w(\varphi)(e_i) = \varphi(w^{-1}(e_i)) \iff w^{-1}(e_i) > 0 \iff w(\varphi) \in A_i^*.\]
    We conclude that \(w(\mathring{C})\subset A_i^*\).
\end{proof}

If we choose in the above lemma \(\Tilde{w} := s_iw\) for an \(i \in I\), we get
\[\ell(s_i\Tilde{w}) = \ell(s_i^2 w) = \ell(w) > \ell(s_iw) = \ell(\Tilde{w}) \iff \Tilde{w} = s_iw(C)\subset A_i^*.\]
This is again equivalent to \(w(\mathring{C}) \subset s_i(A_i^*)\), and we have just proved the following lemma as the counterpart to \Cref{lem:translation}.

\begin{lemma}\label{lem:translation2}
    Let \(w\in W\) and \(i\in I\).
    Then the relation \(\ell(s_iw) < \ell(w)\) holds if and only if we have \(w(\mathring{C})\subset s_i(A_i^*)\), \ie \(\;w\) moves the open chamber into the open halfspace complementary to \(A_i^*\).
\end{lemma}

Building onto this insight, we now want to study the action of parabolic subgroups on the tits cone to get an understanding of the stabilizers of points.
To do so, decompose the fundamental chamber into subsets corresponding to the parabolic subgroups as follows.

\begin{definition}
    Given a parabolic subgroup \(W_J\) corresponding to \(J\subset I\), set
    \[C_J := \bigcap_{j\in J} H_j^* \cap \bigcap_{k\notin J} A_k^*.\]
    We call these the corresponding parabolic subsets (of the fundamental chamber).
\end{definition}

\begin{example}\label{ex:parabolicset} % TODO: add picture
    \begin{enumerate}
        \item When the set \(J\) is empty, the corresponding parabolic subset \(C_\emptyset\) coincides with the entire chamber \(C\).
              Conversely, when \(J\) contains all indices, \(C_J\) reduces to the singleton \(\{0\}\).
        \item If \(J\) is a proper subset of \(I\) with cardinality one, then the corresponding subset \(C_J\) coincides with a codimension-one face of the chamber \(C\).
    \end{enumerate}
\end{example}

\begin{theorem}\label{thm:stabilizer}
    Let \(w\in W\) and \(J, K \subset I\) be subsets.
    Then \(w(C_J)\cap C_K \neq \emptyset\) implies \(J = K\), \(w\in W_J\) and thus \(w(C_J) = C_J\).
    In particular, the isotropy groups of the sets \(C_J\) are the parabolic subgroups \(W_J\).
\end{theorem}
\begin{proof}
    Let \(w\in W\) and \(J, K\subset I\) be subsets, such that \(w(C_J)\cap C_K \neq \emptyset\).
    We also proof this theorem by induction on \(\ell(w)\), with the base case \(\ell(w) = 0\) being trivial.

    Assume \(\ell(w) > 0\) and take \(i\in I\), so that \(\ell(s_iw) < \ell(w)\).
    By writing \(w = s_i(s_iw)\) and applying \Cref{lem:translation2}, we get that \(w(\mathring{C})\subset s_i(A_i^*)\).
    Now we use the continuity of the action of \(W\) to get that \(w(C)\subset \overline{s_i(A_i^*)}\), which then implies that \(w(C)\cap C\subset H_i^*\).
    This follows, since the (closed) fundamental chamber \(C\) by definition is a subset of \(\overline{A_i^*}\).
    Since \(s_i\) fixes \(H_i^*\), it fixes every point in the intersection of \(C\) and its translate \(w(C)\).
    Note that the sets \(C_J\) and \(C_K\) are subsets of the fundamental chamber \(C\) and therefore, \(s_i\) fixes every point in the non-empty set \(w(C_J)\cap C_K\).
    But if \(s_i\) fixes some \(\varphi\in C_K\), then
    \[\varphi(e_i) = s_i(\varphi)(e_i) = \varphi(s_i(e_i)) = -\varphi(e_i) \implies \varphi(e_i) = 0 \iff \varphi\in H_i^*.\]
    We deduce \(i\in K\), respectively \(s_i\in W_K\).
    Using this together with the assumption, we get that \(s_iw(C_J)\cap C_K = s_i(w(C_J)\cap C_K)\) is non-empty.
    We apply the induction hypothesis to the element \(s_iw\), to see that \(J = K\) and \(s_iw\in W_J\).
    Finally, since \(s_i\in W_J = W_K\), we have that \(s_iw \in W_J\) implies \(w\in W_J\), proving the theorem.
\end{proof}

For the next theorem we want to prove, we want to recall the following definition.

\begin{definition}
    Let \(G\) be a group, acting on a topological space \(X\).
    We call a closed subset \(F\subset X\) \emph{strict fundamental domain}, if for each \(x\in F\) its orbit \(Gx\) intersects \(F\) in exactly one point.
\end{definition}

Note that, by definition of the Tits cone \(WC\), every \(W\)-orbit of a point \(\varphi\in C\) meets the fundamental chamber \(C\) in at least one point, namely \(\varphi\). % TODO: check this
Thus, it suffices to proof that each \(W\)-orbit meets \(C\) in at most one point, to prove following theorem:

\begin{theorem}\label{thm:funddomain}
    The fundamental chamber is a fundamental domain for the action on the Tits cone.
\end{theorem}
\begin{proof}
    Assume that \(\varphi,\psi\in C\) lie in the same \(W\)-orbit, but in different sets \(C_J\) and \(C_K\) for \(J,K\subset I\).
    Since they lie in the same orbit, there is a \(w\in W\) with \(\varphi = w(\psi)\).
    Thus, the intersection \(w(C_J)\cap C_K\) is non-empty and \Cref{thm:stabilizer} implies \(J = K\) and \(w\in W_J\).
    We deduce \(\varphi = w(\psi) = \psi\).
    Thus, every \(W\)-orbit of a point \(\varphi\in C\) meets the fundamental chamber \(C\) at most in \(\varphi\), proving the theorem.
\end{proof}

Define a set \(\mathcal{C}\) as the union of all translates of parabolic subsets \(W_J\) \ie, define \(\mathcal{C}\) by
\[\mathcal{C} := \bigcup_{J\subset I}\bigcup_{w\in W} w(C_J).\]
Note that \Cref{thm:stabilizer} implies that this set forms a partition of the Tits cone as it shows that the sets of the form \(w(C_J)\) in \(\mathcal{C}\) are all disjoint for \(w\) in the coset \(W/W_J\) and \(J\) a subset of \(I\).

\begin{theorem}
    The Tits cone \(WC\) is a convex cone, and every closed line segment in the Tits cone meets only finitely many of the sets in \(\mathcal{C}\).
\end{theorem}
\begin{proof}
    First note that the fundamental chamber is a convex cone as the intersection of the finitely many closed halfspaces \(\overline{A_i^*}\).
    This implies that the Tits cone is a cone as well.
    We will prove the convexity by showing that every closed segment between any two points in the Tits cone is contained in it.
    Furthermore, we will prove that these segments can be covered by finitely many of the sets in \(\mathcal{C}\), implying latter statement.

    Consider the closed segment \([\varphi, \psi]\subset V^*\) and assume \(\varphi\in C\) and \(\psi\in w(C)\) for some \(w\in W\), \ie, the endpoints lie in different chambers.
    We proceed by induction on \(\ell(w)\) with the base case \(\ell(w) = 0\) implying \(\varphi,\psi\in C\).
    Since \(C\) is convex and can trivially be covered by finitely many of the \(C_J\), (\eg, set \(J_i = \{s_i\}\) for \(i\in I\)) we are done. % TODO: this eg is wrong
    Therefore, assume \(\ell(w) > 0\).
    Note that we can divide the segment \([\varphi, \psi]\) into two parts, by intersecting it with the closed chamber \(C\) to get a closed segment \([\varphi, \xi]\) inside \(C\) and a segment \([\xi, \psi]\).
    The first segment we can cover by finitely many sets of \(\mathcal{C}\), as it lies in \(C\).
    Thus, we need to show that we can cover the latter segment by finitely many of these sets.
    Assume further, that \(\psi\in s_i(A_i^*)\) for \(i\in J\) and \(\psi\in\overline{A_i^*}\) for \(i\notin J\) for some \(J\subset I\), so that \(\psi\notin C\).
    \Claim{1}{\(\xi\) has to lie in one of the codimension-one faces \(H_i^*\) for \(i\in J\).}
    \Claimproof{1}{
        Assume that \(\xi\in A_i^*\) for \(i\in J\).
        Then every point \(\zeta\) in the intersection of a neighborhood of \(\xi\) contained in \(A_i^*\) and the segment \([\xi,\psi]\) has to also lie in \(A_i^*\) for \(i\in J\). % and \(\zeta\in \overline{A_i^*}\) for \(i\notin J\).
        Clearly, \(\zeta\in\overline{A_i^*}\) for \(i\notin J\) holds as well, implying that \(\zeta\in C\).
        But this is a contradiction to the decomposition of the initial segment \([\varphi, \psi]\).
        Thus, \(\xi\) has to lie in one of the \(H_i^*\), for \(i\in J\).
    }
    Both assumptions together, \(\psi\in s_i(A_i^*)\) and \(\psi\in w(C)\), imply \(w(C)\subset \overline{s_i(A_i^*)}\), hence by continuity \(w(\mathring{C})\subset s_i(A_i^*)\).
    By \Cref{lem:translation2} this is equivalent to \(\ell(s_iw) < \ell(w)\) and we are set up to apply the induction hypothesis to \(\xi\) and \(s_i(\psi)\in s_iw(C)\).
    This gives a finite covering of \([\xi,s_i(\psi)]\) by sets in \(\mathcal{C}\).
    But since we established that \(\xi\in H_i^*\), translation by \(s_i\) gives \([s_i(\xi), s_i^2(\psi)] = [\xi, \psi]\), and thus we can cover this segment by finitely many sets as well.
    The result follows.
\end{proof}

An essential fact we want to emphasize here is that the stabilizer of the fundamental chamber is trivial.
% Setting \(J=K=\emptyset\), we obtain from \Cref{thm:stabilizer} (using that \(C_\emptyset = C\)), that if \(wC\cap C \neq \emptyset\) we must have \(w\in W_\emptyset = \{1\}\).
Setting \(J=\emptyset\), we obtain from \Cref{thm:stabilizer} that the stabilizer subgroup of \(C_J = C_\emptyset = C\) (cf. \Cref{ex:parabolicset}) is precisely the parabolic subgroup \(W_J = W_\emptyset = \{1\}\).


% ***********************************************
\section{Covering actions}
% ***********************************************

% \begin{definition}\label{def:covering}
%     Let \(X, Y\) be topological spaces.
%     A continuous map \(p:Y\to X\) is called a \emph{covering map} and \(Y\) a \emph{covering space} for \(X\), if
%     \begin{itemize}
%         \item \(\forall x\in X: \exists U\in \mathcal{U}_x^X: p^{-1}(U) = \underset{i\in I}{\bigsqcup} U_i\;\) for \(\;U_i\subseteq Y\) open
%         \item \(\forall i\in I:\; p\vert_{U_i} : U_i\to U\) is a homeomorphism
%     \end{itemize}
%     The set \(U\) is then called \emph{evenly covered} and \(\vert I\vert\) the degree of the covering.
%     The sets \(U_i\) are called the \emph{sheets} over \(U\) and for each \(x\in X\), the preimage \(p^{-1}(x)\) is called the \emph{fiber} of \(x\).
% \end{definition}

\begin{definition}\label{def:covering}
    Let \(X, Y\) be topological spaces.
    A continuous map \(p: Y\to X\) is a \emph{covering map} and \(Y\) a \emph{covering space} for \(X\), if every point \(x\) in \(X\) has an open neighborhood \(U\), such that the preimage of \(U\) under \(p\) is a disjoint union of open sets \(U_i\) in \(Y\) for an index set \(I\).
    Furthermore, the map \(p\) has to be a local homeomorphism, meaning the restriction of \(p\) to the \(U_i\) is a homeomorphism.
    Then, \(U\) is called \emph{evenly covered} and \(\vert I \vert\) the degree of the covering, while the open sets \(U_i\) are called the \emph{sheets} over \(U\) and the preimage of an \(x\) in \(X\) is called a \emph{fiber} of \(x\).
\end{definition}

\begin{definition}
    Let \(G\) be a group, acting on a space \(X\).
    If the following is satisfied:
    \[\forall x\in X: \exists U\in \mathcal{U}_x^X: \; \vert\{g\in G\;\vert\; g(U)\cap U \neq \emptyset\}\vert <\infty\]
    we say that the action is \emph{properly discontinuous} on \(X\).
\end{definition}

\begin{definition}
    Let \(G\) be a group, acting on a space \(X\).
    If the more restrictive condition
    \[\forall x\in X: \exists U\in\mathcal{U}_x^X:\; \{g\in G \;\vert\; g(U)\cap U \neq \emptyset\} = \{1\}\]
    is satisfied, we call the action a \emph{covering (space) action}.
\end{definition}

\begin{lemma}
    Let \(G\) be a group, that acts freely and properly discontinuous on a Hausdorff space \(X\).
    Then the action is a covering action in the above sense.
\end{lemma}
\begin{proof}
    Let \(X\) be a Hausdorff space and \(G\) be a group acting freely, properly discontinuously on \(X\).
    Then for a open Neighborhood \(U\) of \(x\in X\) the set \(M := \{g \in G \;\vert\; gU \cap U \neq \emptyset\}\) is finite.
    For these \(g \in M\) we pick pairwise disjoint Neighborhoods \(V_g\) of \(gx\), which is possible since \(G\) acts freely, thus \(gx \neq x\) and \(X\) is Hausdorff.
    Finally, set \(V = \bigcap_{g \in M} g^{-1} V_g \cap U\), which is open as finite intersection of open sets and by definition a Neighborhood of \(x\).
\end{proof}

Note that the Tits cone \(WC\) is a Hausdorff space as a subspace of \(\R^n\) for suitable \(n \in \N\).
Now % TODO: Fulfill this section 

\begin{definition}
    A path-connected topological space \(X\) is called \emph{simply connected}, if \(\pi_1(X)\cong \{1\}\).
\end{definition}

\begin{lemma}
    Let \(G\) be a group, acting by covering action on a simply connected topological space \(X\).
    Then the quotient map \(p_G:(X, x_0)\to (G\backslash X, G\cdot x_0)\) is a covering map and \(\pi_1(G\backslash X, G\cdot x_0)\cong G\).
\end{lemma}
\begin{proof}
    Let \(U\) be an open neighborhood of \(x\in X\), such that \(\{g \in G \;\vert\; gU \cap U \neq \emptyset\} = \{1\}\).
    \Claim{1}{The map \(p_G\) restricted to \(U\) is a continuous bijection.}
    \Claimproof{1}{
        Continuity: \(U\subset G\backslash X\) is open if and only if \(p^{-1}(U)\) is open and thus \(p\) is continuous
        Surjectivity: Since orbits are not empty, \(p\) is surjective
        Injectivity: Assume \(x,y \in X\) with \(p(x) = p(y)\), but then there is  \(g \in G\backslash\{1\}\) with \(gx = y\) implying \(gU \cap U \neq \emptyset\).
        This is a contradiction to \(G\) acting by covering action.
    }\vspace*{-2\parskip}
    \Claim{2}{Every point \(Gx \in G\backslash X\) has a neighborhood \(V \subset G \backslash X\) with \(p_G^{-1}(V) = \bigsqcup_{i \in I} U_i\).}
    \Claimproof{2}{
        By assumption the sets \(gU\) are all disjoint neighborhoods of \(gx \in X\) for all \(g \in G\).
        Moreover, for \(V := p_G(U) \subset G\backslash X\), we have \(p_G^{-1}(V) = \bigsqcup_{g \in G} gU\), proving the Claim.
    }\vspace*{-2\parskip}
    \Claim{3}{The map \(p_G\vert_{gU} \to V = p_G(U)\) is an homeomorphism for all \(g \in G\).}
    \Claimproof{3}{
        Note that the action of \(G\) on \(X\) is by homeomorphisms and \(U \subset X\) is open.
        Then the union \(\bigsqcup_{g \in G} gU\) is open as well and since \(p_G^{-1}(V) = \bigsqcup_{g \in G} gU\), the set \(V\) is open in the quotient topology of \(G \backslash X\).
        Therefore, the map \(p_G\) restricted to \(gU\) for \(g \in G\) is an open map and hence, by \emph{Claim 1} a local homeomorphism.
    }
    Now all conditions in \Cref{def:covering} are satisfied, so \(p_G\) is a covering map.
    For the last part of the statement, take a homotopy class of loops \([\gamma] \in \pi_1(G \backslash X, Gx_0)\) with representative \(\gamma\).
    Define a map \(\varphi : \pi_1(G, x_0) \to G\) on homotopy classes \([\gamma]\), such that \(\varphi([\gamma])\) takes \(\Tilde{\gamma}(0) = x_0\) to \(\Tilde{\gamma}(1)\), where \(\Tilde{\gamma}: [0,1] \to X\) is a lift of \(\gamma\).
    Now if \(\gamma'\) is another representative of \([\gamma]\) then, \(\widetilde{\overline{\gamma}'\cdot \gamma}\) is the lift of a contractible curve in \(G \backslash X\).
\end{proof}
